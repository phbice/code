\documentclass{article}
\usepackage[UTF8]{ctex} %中文支持
\usepackage{geometry}
\usepackage{amsmath} %数学宏包
\usepackage{amssymb}
\usepackage{esint}
\usepackage{bm}
%\usepackage{arcs}
\usepackage{yhmath}
%\usepackage{kpfonts}
\newcommand*{\dif}{\mathop{}\!\mathrm{d}}

\geometry{a4paper, scale = 0.8}
\setlength{\parindent}{0em}
\setlength{\parskip}{.5\baselineskip}
\linespread{1.6}
\title{2010 --- 2017 历届微积分试题整理}
\date{}
\begin{document}
\section*{2010 --- 2011学年第二学期216学时《微积分A2》试题(期终A卷)}

1. 设 $f(x)$ 是周期为2的周期函数, 它在区间$[-1,\,1]$上定义为$f(x)=
\begin{cases}
2.  &-1<x\leqslant0 \\
x^2, &\ \ 0<x\leqslant1
\end{cases}
$\;,
求$f(x)$的傅里叶级数的和函数在$x=2011$处的值. \par

2. 设$\Sigma$为平面$\dfrac{x}{2}+\dfrac{y}{3}+\dfrac{z}{4}=1$在第一卦限中的部分, 
计算曲面积分$\displaystyle{ \iint_{\Sigma}{\left(z+2x+\dfrac{4}{3}y\right) \dif S} }$\;. \par

3. 求曲线$x=t-\sin t, y=1-\cos t, z=4\sin{\dfrac{t}{2}}$
在对应$t=\dfrac{\pi}{2}$点处的法平面方程. \par

4. 计算$\dfrac{1}{2} \displaystyle{\int_0^1{\left[1+\dfrac{x}{2} + \left(\dfrac{x}{2}\right)^2 + \cdots +
\left(\dfrac{x}{2}\right)^n + \cdots \right]\dif x}}$. \par

5. 由方程 $F\!\left(\dfrac{y}{x},\; \dfrac{z}{x}\right)=0$ 确定隐函数$z=z(x,\,y)$
($F$为可微函数), 求$x\dfrac{\partial z}{\partial x} + y\dfrac{\partial z}{\partial y}$. \par

6. 交换二次积分顺序: 
$\displaystyle\int_0^1\mathrm{d}y\displaystyle\int_{\sqrt{y}}^{\sqrt{2-y^2}}f(x,\,y)\dif x$\;. \par

7. 设$\Omega$: $x^2+y^2+z^2\leqslant R^2$ , 计算三重积分: 
$\displaystyle\iiint_{\Omega}{(x^2+2x+y^2+z^2)\dif V}$\par

8. 计算$I=\displaystyle\int_L{(y^3-y\cos x)\dif x - (x^3 + \sin x)\dif y}$ , 
其中$L$为沿$x^2+y^2=1\;(x\geqslant1)$由点$A(0,\,-1)$到点$B(0,\,1)$的弧段. \par

9. 将函数$f(x)=\arctan{\dfrac{1-2x}{1+2x}}$ 展开成$x$的幂级数. \par

10. 设直线$\begin{cases}x+y+b=0\\x+ay-z-3=0\end{cases}$在平面$\pi$上, 
而平面$\pi$与曲面$z=x^2+y^2$相切于点$P(1,\,-2,\,5)$ , 试求常数$a,\,b$. \par

11. 已知曲线积分$\displaystyle\int_L{[xe^{2x}-6f(x)]\sin y \dif x - [5f(x)-f'(x)]\cos y \dif y}$
与路径无关, 其中$f(x)$在$(-\infty, +\infty)$内有二阶连续偏导数, 
且$f(0)=0, \,f'(0)=0$ , 试求$f(x)$ . \par

12. 已知$\Sigma$是空间曲线$\begin{cases}x^2+3y^2=1\\z=0\end{cases}$
绕着$y$轴旋转而成的椭球面, $\Pi$是椭球面$\Sigma$
在第一卦限内的切平面, $S$表示曲面$\Sigma$的上半部分 $(z\geqslant0)$ , 
$\lambda , \mu, v$表示$S$的外法线的方向余弦, 求: \\
(1) $\displaystyle\iint_S{z(\lambda x + 3\mu y + v z)\dif S}$ .\\
(2) 使$\Pi$与三坐标面所围成的四面体的体积最小的切点坐标.

\newpage

\section*{2011 --- 2012学年第二学期216学时《微积分A2》试题(期终A卷)}

1. 设$f(x)$ 是周期为 $2\pi$的周期函数,它在$[-\pi, \pi)$上的表达式为$f(x) = x$, 
试将函数$f(x)$展开成傅里叶级数. \par

2. 求函数$f(x,\,y)=x^2y(4-x-y)$在由直线$x+y=6,\;y=0,\;x=0$所围成的闭区域$D$上的最大值和最小值. \par

3. 判断级数$\displaystyle\sum\limits_{n=1}^\infty{\left[\dfrac{(-1)^n}{\sqrt{n}} + \dfrac{1}{n}\right]}$
的敛散性.\par

4. 设$y=f(x,\,t)$, $t$为由方程$e^t-x y t = 0$确定的$x, y$的函数, 
其中$f$具有一阶连续偏导数, 求$\dfrac{\dif y}{\dif x}$.\par

5. 计算$I = \displaystyle\iiint_\Omega{(x+y)^2\dif V}$, 
其中$\Omega$是由$x^2+y^2=2z$, $z=1$及$z=2$所围成的空间闭区域. \par

6. 设曲面$\Sigma$为柱面$x^2+y^2=1$介于平面$z=1$与$z=2$之间部分的外侧, 
求曲面积分$\displaystyle\iint_\Sigma{(x^2+y^2)}\dif x \dif y$
与$\displaystyle\iint_\Sigma{y^2 \dif S}$. \par

7. 过直线$\begin{cases}10x+2y-2z=27\\ x+y-z=0\end{cases}$
作曲面$3x^2+y^2-z^2=27$的切平面, 求此切平面的方程. \par

8. 求幂级数$\displaystyle\sum\limits_{n=1}^{\infty}{\dfrac{n}{2^n}x^{n-1}}$的收敛域, 和函数以及数项级数$\displaystyle\sum\limits_{n=1}^\infty{(-1)^n\dfrac{n}{2^n}}$的和. \par

9. 设$\,\vec{l}\,$是曲线$\begin{cases}x^2+y^2+z^2=6\\ x+y+z=0\end{cases}$在点$A(1,\,-2,\,1)$
处的切向量, 求函数$f(x,\,y,\,z)=2xy-z^2$在点$A$沿$\,\vec{l}\,$的方向导数, 
并求此函数在点$A$处方向导数的最大值. \par

10. 设函数$f(x)$具有连续导数并满足$f(1)=3$, 计算曲线积分:
$\displaystyle\int_L{(2y f(x) + x)\dif x + (x f(x) + y)\dif y}$的值, 假定此积分在右半平面内与路径无关, 曲线$L$是由$A(1,\,2)$到$B(2,\,1)$的任意一条逐段光滑曲线.\par

11. 计算曲面积分
$I=\displaystyle\iint_\Sigma{2xz^2\dif y \dif z + y(z^2+1)\dif z \dif x + (9-z^2)\dif x \dif y}$, 
其中$\Sigma$为曲面$z=x^2+y^2+1\\(1\leqslant z\leqslant 2)$, 取下侧.\par

\newpage

\section*{2012 --- 2013学年第二学期216学时《微积分A2》试题(期终A卷)}

1. 设$\vec{a}=(3,\,-3,\,4)$, $\vec{b}=(3,\,6,\,2)$, 求$(2\vec{a}+\vec{b}) \bm\cdot (2\vec{a}-\vec{b})$及$(3\vec{a} - 2\vec{b}) \times (5\vec{a} - 4\vec{b})$.\par

2. 求$A, B$ , 使平面$\pi: \; Ax+By+6z-7=0$与直线$l:\; \dfrac{x-4}{2}=\dfrac{y+5}{-4}=\dfrac{z+1}{3}$垂直.\par

3. 设$z=z(x,\,y)$是由方程$x^2-y^2=2[z- \varphi(x+y-z)]$所确定的函数, 其中$\varphi$具有二阶导数, 且$\varphi' \neq 0$. 求: \\
(1) $\dif z$; (2) 记$u(x,\,y)=\dfrac{1}{x+y}\left(\dfrac{\partial z}{\partial x} - \dfrac{\partial z}{\partial y}\right)$, 求$\dfrac{\partial u}{\partial x}$. \par

4. 计算二重积分$\displaystyle\iint_D{|y| \dif x \dif y}$, 其中$D: \dfrac{x^2}{a^2}+\dfrac{y^2}{b^2} \leqslant 1$. \par

5. 计算三重积分$\displaystyle\iiint_\Omega{z \dif V}$, 其中$\Omega$是由锥面$\sqrt{x^2+y^2}=z$及平面$z=1, z=2$所围成的闭区域. \par

6. 已知$\displaystyle\int_C{\varphi(x) y \dif y + x y^2[\varphi(x)+1] \dif x}$在全平面上与路径无关, 其中$\varphi(x)$具有连续的一阶导数, 并当$C$是起点在$(0,\,0)$, 终点在$(1,\,1)$的有向曲线时, 该曲线积分值为$\dfrac{1}{2}$, 试求函数$\varphi(x)$. \par

7. 计算曲面积分$\displaystyle\iint_\Sigma{x y + y z + z}$, 其中$\Sigma$为锥面$z=\sqrt{x^2+y^2}$被柱面$x^2+y^2 = 2a x \; (a>0)$所截得的有限部分. \par

8. 求曲面$z=\dfrac{x^2}{2}+y^2$平行于平面$2x+2y-z=0$的切平面方程.\par

9. 试求幂级数$\displaystyle\sum\limits_{n=1}^\infty{\dfrac{(-1)^{n-1}}{n\cdot 6^n}x^{n+1}}$的收敛区间及和函数$S(x)$.\par

10. 求曲面积分$\displaystyle\iint_S{2 x^3 \dif y \dif z + 2 y^3 \dif z \dif x + 3(z^2-1) \dif x \dif y}$, 其中$S$是由曲线$\begin{cases}z=1-y^2\\ x=0\end{cases}$绕$z$轴旋转一周而成的旋转面$(z\geqslant0)$的上侧.\par

11. 试在曲面$S: 2 x^2 + y^2 + z^2 = 1$上求一点, 使得函数$f(x,\,y,\,z)=x^2+y^2+z^2$沿着点$A(1,\,1,\,1)$到点$B(2,\,0,\,1)$的方向导数具有最大值.\par

12. 设函数$f(x)$在$(-\infty,\,+\infty)$上有定义, 在$x=0$的某个邻域内有一阶连续导数且$\lim\limits_{x \to 0}{\dfrac{f(x)}{x}} = a > 0$, 证明: $\displaystyle\sum\limits_{n=1}^\infty{(-1)^n f\left(\dfrac{1}{n}\right)}$收敛, 而$\displaystyle\sum\limits_{n=1}^\infty{f\left(\dfrac{1}{n}\right)}$发散.

\newpage

\section*{2013 --- 2014学年第二学期216学时《微积分A2》试题(期终A卷)}
1. 利用二重积分的性质, 比较积分$I_1 = \displaystyle\iint_D{\ln(x^2+y^2) \dif \sigma}$与$I_2 = \displaystyle\iint_D|\ln(x^2+y^2)|^2 \dif \sigma$的大小, 其中$D: e\leqslant x^2+y^2\leqslant2e$.\par

2. 设$z = f\left(xy,\,\dfrac{x}{y}\right)+\sin y$, 其中$f$具有二阶连续偏导数, 求$\dfrac{\partial z}{\partial x},\; \dfrac{\partial^2 z}{\partial x \partial y}$.\par

3. 求过点$M(1, -2, 3)$的平面, 使它与平面$\pi:\;x+y-z-3=0$垂直, 且与直线$L: x=y=z$平行.\par

4. 设函数$z=z(x,\, y)$由方程$z\sin(x+y) = y+z$确定, 求全微分 $\dif z$.\par

5. 计算曲线积分$\displaystyle\int_L{(2a-y) \dif x + x \dif y}$, 式中$L$是从原点$(0,\,0)$沿曲线$\begin{cases}x=a(t-\sin t) \\ y=a(1-\cos t) \end{cases}(a>0)$ 到点$A(2\pi a, \, 0)$的弧段.\par

6. 求二元函数$f(x,\,y)=e x - \dfrac{x^2+y^2}{2}$的极值.\par

7. 计算曲面积分$\displaystyle\oiint_\Sigma{(y+z-x) \dif y \dif z + (x+y -z) \dif z \dif x + (x+y+z) \dif x \dif y}$, 其中$\Sigma$是由$|x|\leqslant 1,\; |y|\leqslant 1,\; |z|\leqslant 1$所确定的立体的表面外侧.\par

8. 求曲线$x=\sin^2 x, y=\sin t \cdot \cos t, z=\cos^2 t$在对应于$t=\dfrac{\pi}{4}$的点的切线和法平面方程.\par

9. 设$\Omega$是由曲面$z=x^2+y^2,\; z=2-\sqrt{x^2+y^2}$所围的有界区域, 试计算$I=\displaystyle\iiint_\Omega{(x^2+y^2)(\sin x + 1) \dif V}$.\par

10. 设$h(x)=\begin{cases} \dfrac{\ln(1+x)}{x}, &x>-1,\;x \neq 0 \\ 1, &x=0\end{cases}$, 试将$f(x)=\displaystyle\int_0^x{h(t) \dif t}$展开成$x$的幂级数.\par

11. 试求锥面$x^2+y^2=\dfrac{16}{9}z^2$被柱面$(x-2)^2+y^2=4$截下部分的面积.\par

12. 设$f(x)=\begin{cases}x+2\pi, &-\pi\leqslant x < 0 \\ \pi, &x=0\\ x, &0 < x \leqslant \pi \end{cases},\;a_n=\dfrac{1}{\pi}\displaystyle\int_{-\pi}^{\pi}{f(x)\cos nx \dif x},\; n=0,\,1,\,2,\cdots$, 利用函数的傅里叶级数展开式, 求数项级数$\displaystyle\sum\limits_{n=0}^\infty {a_n}$的和.\par

13. 设$\Omega$为区域$x^2+y^2+z^2\leqslant 1$, $P(x_0,\,y_0,\,z_0)$为$\Omega$外的一点, 试证:
$$\iiint_\Omega{e^{-\sqrt{(x-x_0)^2+(y-y_0)^2+(z-z_0)^2}} \dif V} \geqslant \frac{4\pi}{3}e^{-(1+x_0^2+y_0^2+z_0^2)}$$

\newpage

\section*{2014 --- 2015学年第二学期216学时《微积分A2》试题(期终A卷)}
1. 设$\vec{p}=2\vec{a}+\vec{b},\; \vec{q}=k\vec{a}+\vec{b}$, 其中$|\vec{a}|=1,\; |\vec{b}|=2$, 且$\vec{a}\,\bot\, \vec{b}$, 问: (1) $k$为何值时, $\vec{p} \,\bot\, \vec{q}\;$? (2) $k$为何值时, 以$\vec{p},\; \vec{q}$为边的平行四边形面积为6?\par

2. 设有数量场$u=\dfrac{x^2}{a^2}+\dfrac{y^2}{b^2}-\dfrac{z^2}{c^2}$, 问: $a,\,b,\,c$满足什么条件, 才能使函数$u(x,\,y,\,z)$在点$p(1,\,-2,\,2)$处沿方向$\vec{l}=(1,\,-2,\,-1)$的方向导数最大?\par

3. 函数$z=z(x,\,y)$由方程$z=f(x+y+z)$所确定, 其中$f$二阶可导, 且$f'(u)\neq1$, 求$\dfrac{\partial^2 z}{\partial x^2}$.\par

4. 设$u=f(x+y+z, x y z)$具有一阶连续偏导数, 其中$z=z(x,\, y)$由方程$x^2+2z e^{y^2}=\sin z$所确定, 求$\dif z$.\par

5. 求曲线$\begin{cases}ax+by+cz=ab+2bc \\ a^2x^2-b^2y^2+c^2z^2=a^2b^2\end{cases}$在点$(b,\,c,\,b)$处的切线及法平面方程(其中$a\neq0,\, b\neq0,\, c\neq0$).\par

6. 设$L_1:\,\dfrac{x+2}{1}=\dfrac{y-3}{-1}=\dfrac{z+1}{1},\; L_2:\, \dfrac{x+4}{2}=\dfrac{y}{1}=\dfrac{z-4}{3}$, 试求与直线$L_1,\,L_2$都垂直且相交的直线的方程.\par

7. 设$z=x^3 + \alpha x^2 + 2\gamma xy + \beta y^2 + \alpha \beta^{-2}(\gamma x + \beta y)$, 试证: 当$\alpha \beta \neq \gamma^2$时, 函数$z$有一个且仅有一个极值. 又若$\beta < 0$, 则该极值必为极大值.\par

8. 计算$\displaystyle\oiint_\Sigma{x^2 \dif y \dif z - y^2 \dif z \dif x + z^2 \dif x \dif y}$, 其中$\Sigma$是立体$\Omega$的表面的外侧, 立体$\Omega$由球面$x^2+y^2+z^2=3R^2$与单叶双曲面$x^2+y^2-z^2=R^2$及平面$z=0$所围成的含有$Oz$轴正半轴的那部分.\par

9. 确定常数$\lambda$, 使得在不包含$x$轴的单连域内,曲线积分
$$\int_L{\frac{x}{y}(x^2+2xy+2y^2) \dif x -\frac{x^2}{y^2}(x^2+2xy+2y^2)^2 \dif y} = \int_L{P \dif x + Q\dif y}$$
与路径无关, 并在上述条件下, 求$\displaystyle\int_{(-3,3)}^{(-1,1)}{P \dif x + Q \dif y}$.\par

10. 求幂级数$\displaystyle\sum\limits_{n=1}^\infty{(2n+1)\dfrac{x^{n-1}}{3^n}}$在$(-3,\,3)$内的和函数.\par

11. 设幂级数$f(x)=\displaystyle\sum\limits_{n=0}^\infty{a_nx^n}$在$[0,\,1]$上收敛, 证明: 当$a_0=a_1=0$时, 数项级数$\displaystyle\sum\limits_{n=1}^\infty{f\left(\dfrac{1}{n}\right)}$收敛.\par

12. 设函数$f(x,\,y)$在单位圆上有连续的偏导数, 且在边界上取值为零, 计算极限:
$$\lim_{\varepsilon \to 0}{\frac{-1}{2\pi}\iint_D{\frac{x\frac{\partial f}{\partial x}+y\frac{\partial f}{\partial y}}{x^2+y^2} \dif x \dif y}}$$
其中$D$为圆环域: $\varepsilon^2\leqslant x^2+y^2\leqslant 1$

\newpage

\section*{2015 --- 2016学年第二学期216学时《微积分A2》试题(期终A卷)}
1. 设长方体三条棱长为$|OA|=5,\;|OB|=3,\;|OC|=4$, $OM$为对角线, 求$\overrightarrow{OA}$在$\overrightarrow{OM}$上的投影.\par

2. 求由$x^3+y^3+z^3-3axyz = 0$确定的隐函数$z = z(x,\,y)$在点$(0,\,-4)$处沿$\vec{a}=(7,\,-9)$方向的方向导数.\par

3. 设$z=z(x,\,y)$由方程$z=x+y\varphi(z)$确定, 其中$\varphi$二阶可导, 且$1-y\varphi'(z) \neq 0$, 求$\dfrac{\partial^2 z}{\partial x^2}$.\par

4. 求函数$z=2x^2+3y^2+4x-8$在闭域$D: x^2+y^2 \leqslant 4$上的最大值和最小值.\par

5. 设$\Omega$是由$z=\sqrt{1-x^2-y^2}$及$z=0$所围的闭区域, 试将$\displaystyle\iiint\limits_\Omega{f(x^2+y^2) \dif V}$分别化为球面, 柱面坐标下的三次积分式.\par

6. 求二元可微函数$\varphi(x,\,y)$, 满足$\varphi(0,\,1) = 1$, 并使曲线积分$I_1= \displaystyle\int_L{(3x y^2 +x^3)\dif x + \varphi(x,\,y) \dif y}$ 及\\
$I_2= \displaystyle\int_L{\varphi(x,\,y) \dif x+(3x y^2 + x^3) \dif y}$ 都与积分路径无关.\par

7. 设$f(x)$是以$2\pi$为周期的分段连续函数, 已知$f(x)$的傅里叶系数为$a_0,\,a_n,\,b_n,\;n=1,\,2,\,3,\cdots$,  试用$a_0,\,a_n,\,b_n$表示函数$\varphi(x) = -f(-x)$的傅里叶系数$A_0,\,A_n,\,B_n,\;n=1,\,2,\,3,\cdots$.\par

8. 求曲线$\begin{cases}y^2+z^2=20 \\ -2x^2 -2y^2+z^2 = -10\end{cases}$在点$(-3,\,2,\,4)$处的切线及法平面方程.\par

9. 计算曲面积分$I= \displaystyle\oiint\limits_\Sigma{xy \dif x \dif z + xy^2 \dif y \dif z}$, 其中$\Sigma$是曲面$x=y^2+z^2$与平面$y=0,\,z=0,\,x=1$在第一卦限所围成立体的表面的内侧.\par

10. 试求函数$f(x)=\arctan x$在点$x_0 = 0$的泰勒级数展开式, 并求$\displaystyle\sum\limits_{n=1}^\infty{\dfrac{(-1)^n}{2n-1}\left(\dfrac{3}{4}\right)^n}$.\par

11. 设有直线$l_1:\, \dfrac{x}{4}=\dfrac{y}{1}=\dfrac{z}{1}, \; l_2:\, \begin{cases}z=5x-6 \\ z=4y+3\end{cases},\; l_3:\, \begin{cases}y=2x-4 \\ z=3y+5\end{cases}$, 求平行于$l_1$而分别于$l_2,\,l_3$相交的直线的方程.\par

12. 若级数$\displaystyle\sum\limits_{n=1}^\infty{a_n}\;(a_n>0)$收敛, 试证明: 当$\alpha > 1$时, $\displaystyle\sum\limits_{n=1}^\infty{\sqrt{\dfrac{a_n}{n^\alpha}}}$也收敛.\par

\newpage

\section*{2016 --- 2017学年第二学期216学时《微积分A2》试题(期终A卷)}
1. 设$|\vec{a}|=2,\, |\vec{b}|=3$, 求$|\vec{a} \times \vec{b}|^2 + (\vec{a} \bm\cdot \vec{b})^2$.\par

2. 过直线$l:\, \begin{cases}x+y-z=0 \\ x+2y+z=0 \end{cases}$作两个相互垂直的平面, 且其中一个过已知点$M(0,\,1,\,-1)$, 求这两个平面的方程.\par

3. 讨论极限$\lim\limits_{\substack{x \to 0 \\ y \to 0}}\dfrac{x^4y^4}{(x^2+y^4)^3}$的存在性, 若存在求出极限, 若不存在说明理由.\par

4. 设$z=f(u,\,v)$具有二阶连续偏导数, 其中$u=e^x\cos y, \; v=e^x \sin y$, 试证明: 若$\dfrac{\partial^2f}{\partial u^2}
+\dfrac{\partial^2f}{\partial v^2} = 0$, 则$\dfrac{\partial^2z}{\partial x^2}+\dfrac{\partial^2z}{\partial y^2}=0$.\par

5. 设$z=f(\sqrt{u^2+v^2})$, 其中$f$具有连续导数, $u=e^y\sin x, \; v=e^x\cos y$, 求$\dif z$.\par

6. 在椭球面$2x^2+2y^2+z^2=1$上求一点, 使函数$f(x,\,y,\,z)=x^2+y^2+z^2$在该点沿方向$\vec{l}=\vec{i}-\vec{j}$的方向导数最大.\par

7. 设$y=\varphi(x) \;\; (\,x \in [1,\,3]\,)$ 是具有连续导数的函数, 点$A(1,\,2)$及点$B(3,\,4)$在曲线$L: y=\varphi(x)$上, 而$L$恒在弦$\overline{BA}$之上, 且弧$\wideparen{AB}$与$\overline{BA}$所围成弓形$D$的面积为$S$, 试计算曲线积分
$\displaystyle\int_{\wideparen{AB}}{\dfrac{y}{x^2}\dif x + (x- \dfrac{1}{x})\dif y}$.\par

8. 设$f(x)$是以$2\pi$为周期的可微周期函数, 又设$f'(x)$连续, $a_0, \, a_n, \, b_n \, (n= 1, \, 2, \, 3, \cdots)$是
$f(x)$的傅里叶系数. 求证: $\lim\limits_{n \to \infty}{a_n=0}, \; \lim\limits_{n \to \infty}{b_n = 0}$.\par

9. 试将函数$f(x)=\ln(1+x+x^2)$展开成$x$的幂级数.\par

10. 设有向量场$\vec{F} = \left(x^2yz^2, \quad \dfrac{1}{z}\arctan \dfrac{y}{z} - xy^2z^2, \quad \dfrac{1}{y}\arctan \dfrac{y}{z} + z(1+xyz)\right)$.\\
(1) 计算$\left.\mathrm{div} \vec{F} \right|_{(1,\,1,\,1)}$的值.\\
(2)设空间区域$\Omega$由锥面$y^2+z^2=x^2$与球面$x^2+y^2+z^2=a^2, \; x^2+y^2+z^2=4a^2$所围成$(x>0)$, 其中$a$为正常数, 
记$\Omega$表面的外侧为$\Sigma$, 计算积分
$$I=\oiint\limits_\Sigma{\vec{F} \bm\cdot \dif \vec{S}} = \oiint\limits_\Sigma{x^2yz^2 \dif y \dif z +
\left[\frac{1}{z}\arctan\frac{y}{z} + xy^2\right] \dif z \dif x +
\left[\frac{1}{y}\arctan\frac{y}{z} + z(1+xyz)\right] \dif x \dif y}$$
\par

11. 设$L$是正方形域$D: 0\leqslant x \leqslant 1, \; 0\leqslant y \leqslant 1$的正向边界, $f(x)$是正值连续函数, 试证:
$$I=\oint_L{x f(y) \dif y - \dfrac{y}{f(x)} \dif x} \geqslant 2$$\par

\end{document}
